% Options for packages loaded elsewhere
\PassOptionsToPackage{unicode}{hyperref}
\PassOptionsToPackage{hyphens}{url}
%
\documentclass[
]{article}
\usepackage{lmodern}
\usepackage{amssymb,amsmath}
\usepackage{ifxetex,ifluatex}
\ifnum 0\ifxetex 1\fi\ifluatex 1\fi=0 % if pdftex
  \usepackage[T1]{fontenc}
  \usepackage[utf8]{inputenc}
  \usepackage{textcomp} % provide euro and other symbols
\else % if luatex or xetex
  \usepackage{unicode-math}
  \defaultfontfeatures{Scale=MatchLowercase}
  \defaultfontfeatures[\rmfamily]{Ligatures=TeX,Scale=1}
\fi
% Use upquote if available, for straight quotes in verbatim environments
\IfFileExists{upquote.sty}{\usepackage{upquote}}{}
\IfFileExists{microtype.sty}{% use microtype if available
  \usepackage[]{microtype}
  \UseMicrotypeSet[protrusion]{basicmath} % disable protrusion for tt fonts
}{}
\makeatletter
\@ifundefined{KOMAClassName}{% if non-KOMA class
  \IfFileExists{parskip.sty}{%
    \usepackage{parskip}
  }{% else
    \setlength{\parindent}{0pt}
    \setlength{\parskip}{6pt plus 2pt minus 1pt}}
}{% if KOMA class
  \KOMAoptions{parskip=half}}
\makeatother
\usepackage{xcolor}
\IfFileExists{xurl.sty}{\usepackage{xurl}}{} % add URL line breaks if available
\IfFileExists{bookmark.sty}{\usepackage{bookmark}}{\usepackage{hyperref}}
\hypersetup{
  pdftitle={INLArethinking\_HW4},
  pdfauthor={Ania Kawiecki},
  hidelinks,
  pdfcreator={LaTeX via pandoc}}
\urlstyle{same} % disable monospaced font for URLs
\usepackage[margin=1in]{geometry}
\usepackage{color}
\usepackage{fancyvrb}
\newcommand{\VerbBar}{|}
\newcommand{\VERB}{\Verb[commandchars=\\\{\}]}
\DefineVerbatimEnvironment{Highlighting}{Verbatim}{commandchars=\\\{\}}
% Add ',fontsize=\small' for more characters per line
\usepackage{framed}
\definecolor{shadecolor}{RGB}{248,248,248}
\newenvironment{Shaded}{\begin{snugshade}}{\end{snugshade}}
\newcommand{\AlertTok}[1]{\textcolor[rgb]{0.94,0.16,0.16}{#1}}
\newcommand{\AnnotationTok}[1]{\textcolor[rgb]{0.56,0.35,0.01}{\textbf{\textit{#1}}}}
\newcommand{\AttributeTok}[1]{\textcolor[rgb]{0.77,0.63,0.00}{#1}}
\newcommand{\BaseNTok}[1]{\textcolor[rgb]{0.00,0.00,0.81}{#1}}
\newcommand{\BuiltInTok}[1]{#1}
\newcommand{\CharTok}[1]{\textcolor[rgb]{0.31,0.60,0.02}{#1}}
\newcommand{\CommentTok}[1]{\textcolor[rgb]{0.56,0.35,0.01}{\textit{#1}}}
\newcommand{\CommentVarTok}[1]{\textcolor[rgb]{0.56,0.35,0.01}{\textbf{\textit{#1}}}}
\newcommand{\ConstantTok}[1]{\textcolor[rgb]{0.00,0.00,0.00}{#1}}
\newcommand{\ControlFlowTok}[1]{\textcolor[rgb]{0.13,0.29,0.53}{\textbf{#1}}}
\newcommand{\DataTypeTok}[1]{\textcolor[rgb]{0.13,0.29,0.53}{#1}}
\newcommand{\DecValTok}[1]{\textcolor[rgb]{0.00,0.00,0.81}{#1}}
\newcommand{\DocumentationTok}[1]{\textcolor[rgb]{0.56,0.35,0.01}{\textbf{\textit{#1}}}}
\newcommand{\ErrorTok}[1]{\textcolor[rgb]{0.64,0.00,0.00}{\textbf{#1}}}
\newcommand{\ExtensionTok}[1]{#1}
\newcommand{\FloatTok}[1]{\textcolor[rgb]{0.00,0.00,0.81}{#1}}
\newcommand{\FunctionTok}[1]{\textcolor[rgb]{0.00,0.00,0.00}{#1}}
\newcommand{\ImportTok}[1]{#1}
\newcommand{\InformationTok}[1]{\textcolor[rgb]{0.56,0.35,0.01}{\textbf{\textit{#1}}}}
\newcommand{\KeywordTok}[1]{\textcolor[rgb]{0.13,0.29,0.53}{\textbf{#1}}}
\newcommand{\NormalTok}[1]{#1}
\newcommand{\OperatorTok}[1]{\textcolor[rgb]{0.81,0.36,0.00}{\textbf{#1}}}
\newcommand{\OtherTok}[1]{\textcolor[rgb]{0.56,0.35,0.01}{#1}}
\newcommand{\PreprocessorTok}[1]{\textcolor[rgb]{0.56,0.35,0.01}{\textit{#1}}}
\newcommand{\RegionMarkerTok}[1]{#1}
\newcommand{\SpecialCharTok}[1]{\textcolor[rgb]{0.00,0.00,0.00}{#1}}
\newcommand{\SpecialStringTok}[1]{\textcolor[rgb]{0.31,0.60,0.02}{#1}}
\newcommand{\StringTok}[1]{\textcolor[rgb]{0.31,0.60,0.02}{#1}}
\newcommand{\VariableTok}[1]{\textcolor[rgb]{0.00,0.00,0.00}{#1}}
\newcommand{\VerbatimStringTok}[1]{\textcolor[rgb]{0.31,0.60,0.02}{#1}}
\newcommand{\WarningTok}[1]{\textcolor[rgb]{0.56,0.35,0.01}{\textbf{\textit{#1}}}}
\usepackage{graphicx,grffile}
\makeatletter
\def\maxwidth{\ifdim\Gin@nat@width>\linewidth\linewidth\else\Gin@nat@width\fi}
\def\maxheight{\ifdim\Gin@nat@height>\textheight\textheight\else\Gin@nat@height\fi}
\makeatother
% Scale images if necessary, so that they will not overflow the page
% margins by default, and it is still possible to overwrite the defaults
% using explicit options in \includegraphics[width, height, ...]{}
\setkeys{Gin}{width=\maxwidth,height=\maxheight,keepaspectratio}
% Set default figure placement to htbp
\makeatletter
\def\fps@figure{htbp}
\makeatother
\setlength{\emergencystretch}{3em} % prevent overfull lines
\providecommand{\tightlist}{%
  \setlength{\itemsep}{0pt}\setlength{\parskip}{0pt}}
\setcounter{secnumdepth}{-\maxdimen} % remove section numbering

\title{INLArethinking\_HW4}
\author{Ania Kawiecki}
\date{8/3/2020}

\begin{document}
\maketitle

\begin{itemize}
\tightlist
\item
  {[}Statistical rethinking homework solutions{]}
  (\url{https://github.com/rmcelreath/statrethinking_winter2019/tree/master/homework})
\end{itemize}

INLA book:
\url{https://becarioprecario.bitbucket.io/inla-gitbook/ch-intro.html}

\begin{Shaded}
\begin{Highlighting}[]
\KeywordTok{library}\NormalTok{(tidyverse)}
\KeywordTok{library}\NormalTok{(rethinking)}
\KeywordTok{library}\NormalTok{(dagitty)}
\KeywordTok{library}\NormalTok{(INLA)}
\end{Highlighting}
\end{Shaded}

\hypertarget{homework-4}{%
\section{HOMEWORK 4}\label{homework-4}}

\hypertarget{consider-three-fictional-polynesian-islands.-on-each-there-is-a-royal-or--nithologist-charged-by-the-king-with-surveying-the-bird-population.-they-have-each-found-the-following-proportions-of-5-important-birb-species}{%
\subsection{1. Consider three fictional Polynesian islands. On each
there is a Royal Or- nithologist charged by the king with surveying the
bird population. They have each found the following proportions of 5
important birb
species:}\label{consider-three-fictional-polynesian-islands.-on-each-there-is-a-royal-or--nithologist-charged-by-the-king-with-surveying-the-bird-population.-they-have-each-found-the-following-proportions-of-5-important-birb-species}}

Notice that each row sums to 1, all the birbs. This problem has two
parts. It is not computationally complicated. But it is conceptually
tricky.

First, compute the entropy of each island's birb distribution. Interpret
these entropy values.

Second, use each island's birb distribution to predict the other two.
This means to compute the K-L Divergence of each island from the others,
treat- ing each island as if it were a statistical model of the other
islands. You should end up with 6 different K-L Divergence values. Which
island predicts the others best? Why?

\begin{Shaded}
\begin{Highlighting}[]
\NormalTok{H <-}\StringTok{ }\ControlFlowTok{function}\NormalTok{(p) }\OperatorTok{-}\KeywordTok{sum}\NormalTok{(p}\OperatorTok{*}\KeywordTok{log}\NormalTok{(p))}

\NormalTok{IB <-}\StringTok{ }\KeywordTok{list}\NormalTok{()}
\NormalTok{IB[[}\DecValTok{1}\NormalTok{]] <-}\StringTok{ }\KeywordTok{c}\NormalTok{( }\FloatTok{0.2}\NormalTok{ , }\FloatTok{0.2}\NormalTok{ , }\FloatTok{0.2}\NormalTok{ , }\FloatTok{0.2}\NormalTok{ , }\FloatTok{0.2}\NormalTok{ )}
\NormalTok{IB[[}\DecValTok{2}\NormalTok{]] <-}\StringTok{ }\KeywordTok{c}\NormalTok{( }\FloatTok{0.8}\NormalTok{ , }\FloatTok{0.1}\NormalTok{ , }\FloatTok{0.05}\NormalTok{ , }\FloatTok{0.025}\NormalTok{ , }\FloatTok{0.025}\NormalTok{ )}
\NormalTok{IB[[}\DecValTok{3}\NormalTok{]] <-}\StringTok{ }\KeywordTok{c}\NormalTok{( }\FloatTok{0.05}\NormalTok{ , }\FloatTok{0.15}\NormalTok{ , }\FloatTok{0.7}\NormalTok{ , }\FloatTok{0.05}\NormalTok{ , }\FloatTok{0.05}\NormalTok{ )}
\KeywordTok{sapply}\NormalTok{( IB , H )}
\end{Highlighting}
\end{Shaded}

\begin{verbatim}
## [1] 1.6094379 0.7430039 0.9836003
\end{verbatim}

The first island has the largest entropy, followed by the third, and
then the second in last place. Why is this? Entropy is a measure of the
evenness of a distribution. The first islands has the most even
distribution of birbs. This means you wouldn't be very surprised by any
particular birb. The second island, in contrast, has a very uneven
distribution of birbs. If you saw any birb other than the first species,
it would be surprising.

\begin{Shaded}
\begin{Highlighting}[]
\NormalTok{DKL <-}\StringTok{ }\ControlFlowTok{function}\NormalTok{(p,q) }\KeywordTok{sum}\NormalTok{( p}\OperatorTok{*}\NormalTok{(}\KeywordTok{log}\NormalTok{(p)}\OperatorTok{-}\KeywordTok{log}\NormalTok{(q)) )}



\NormalTok{Dm <-}\StringTok{ }\KeywordTok{matrix}\NormalTok{( }\OtherTok{NA}\NormalTok{ , }\DataTypeTok{nrow=}\DecValTok{3}\NormalTok{ , }\DataTypeTok{ncol=}\DecValTok{3}\NormalTok{ )}

\ControlFlowTok{for}\NormalTok{ ( i }\ControlFlowTok{in} \DecValTok{1}\OperatorTok{:}\DecValTok{3}\NormalTok{ ) }\ControlFlowTok{for}\NormalTok{ ( j }\ControlFlowTok{in} \DecValTok{1}\OperatorTok{:}\DecValTok{3}\NormalTok{ ) Dm[i,j] <-}\StringTok{ }\KeywordTok{DKL}\NormalTok{( IB[[j]] , IB[[i]])}

\NormalTok{test <-}\StringTok{ }\KeywordTok{DKL}\NormalTok{( IB[[}\DecValTok{1}\NormalTok{]] , IB[[}\DecValTok{2}\NormalTok{]])}
  
\KeywordTok{round}\NormalTok{( Dm , }\DecValTok{2}\NormalTok{ )}
\end{Highlighting}
\end{Shaded}

\begin{verbatim}
##      [,1] [,2] [,3]
## [1,] 0.00 0.87 0.63
## [2,] 0.97 0.00 1.84
## [3,] 0.64 2.01 0.00
\end{verbatim}

\begin{enumerate}
\def\labelenumi{\arabic{enumi}.}
\setcounter{enumi}{1}
\tightlist
\item
  Recall them arriage,age,and happiness collider bias example from
  Chapter 6. Run models m6.9 and m6.10 again. Compare these two models
  using WAIC (or LOO, they will produce identical results). Which model
  is ex- pected to make better predictions? Which model provides the
  correct causal inference about the influence of age on happiness? Can
  you explain why the answers to these two questions disagree?
\end{enumerate}

\begin{Shaded}
\begin{Highlighting}[]
\KeywordTok{library}\NormalTok{(rethinking)}
\NormalTok{d <-}\StringTok{ }\KeywordTok{sim_happiness}\NormalTok{( }\DataTypeTok{seed=}\DecValTok{1977}\NormalTok{ , }\DataTypeTok{N_years=}\DecValTok{1000}\NormalTok{ )}
\KeywordTok{precis}\NormalTok{(d)}
\end{Highlighting}
\end{Shaded}

\begin{verbatim}
##                    mean        sd      5.5%     94.5%     histogram
## age        3.300000e+01 18.768883  4.000000 62.000000 ▇▇▇▇▇▇▇▇▇▇▇▇▇
## married    3.007692e-01  0.458769  0.000000  1.000000    ▇▁▁▁▁▁▁▁▁▃
## happiness -1.000070e-16  1.214421 -1.789474  1.789474      ▇▅▇▅▅▇▅▇
\end{verbatim}

\begin{Shaded}
\begin{Highlighting}[]
\NormalTok{d2 <-}\StringTok{ }\NormalTok{d[ d}\OperatorTok{$}\NormalTok{age}\OperatorTok{>}\DecValTok{17}\NormalTok{ , ] }\CommentTok{# only adults}
\NormalTok{d2}\OperatorTok{$}\NormalTok{A <-}\StringTok{ }\NormalTok{( d2}\OperatorTok{$}\NormalTok{age }\OperatorTok{-}\StringTok{ }\DecValTok{18}\NormalTok{ ) }\OperatorTok{/}\StringTok{ }\NormalTok{( }\DecValTok{65} \OperatorTok{-}\StringTok{ }\DecValTok{18}\NormalTok{ )}
\NormalTok{d2}\OperatorTok{$}\NormalTok{mid <-}\StringTok{ }\NormalTok{d2}\OperatorTok{$}\NormalTok{married }\OperatorTok{+}\StringTok{ }\DecValTok{1}

\CommentTok{#contains both marriage status and age}
\NormalTok{m6}\FloatTok{.9}\NormalTok{ <-}\StringTok{ }\KeywordTok{quap}\NormalTok{(}
    \KeywordTok{alist}\NormalTok{(}
\NormalTok{        happiness }\OperatorTok{~}\StringTok{ }\KeywordTok{dnorm}\NormalTok{( mu , sigma ),}
\NormalTok{        mu <-}\StringTok{ }\NormalTok{a[mid] }\OperatorTok{+}\StringTok{ }\NormalTok{bA}\OperatorTok{*}\NormalTok{A,}
\NormalTok{        a[mid] }\OperatorTok{~}\StringTok{ }\KeywordTok{dnorm}\NormalTok{( }\DecValTok{0}\NormalTok{ , }\DecValTok{1}\NormalTok{ ),}
\NormalTok{        bA }\OperatorTok{~}\StringTok{ }\KeywordTok{dnorm}\NormalTok{( }\DecValTok{0}\NormalTok{ , }\DecValTok{2}\NormalTok{ ),}
\NormalTok{        sigma }\OperatorTok{~}\StringTok{ }\KeywordTok{dexp}\NormalTok{(}\DecValTok{1}\NormalTok{)}
\NormalTok{    ) , }\DataTypeTok{data=}\NormalTok{d2 )}
\KeywordTok{precis}\NormalTok{(m6}\FloatTok{.9}\NormalTok{,}\DataTypeTok{depth=}\DecValTok{2}\NormalTok{)}
\end{Highlighting}
\end{Shaded}

\begin{verbatim}
##             mean         sd       5.5%      94.5%
## a[1]  -0.2350877 0.06348986 -0.3365568 -0.1336186
## a[2]   1.2585517 0.08495989  1.1227694  1.3943340
## bA    -0.7490274 0.11320112 -0.9299447 -0.5681102
## sigma  0.9897080 0.02255800  0.9536559  1.0257600
\end{verbatim}

\begin{Shaded}
\begin{Highlighting}[]
\CommentTok{#this model to a model that omits marriage status.}
\NormalTok{m6}\FloatTok{.10}\NormalTok{ <-}\StringTok{ }\KeywordTok{quap}\NormalTok{(}
    \KeywordTok{alist}\NormalTok{(}
\NormalTok{        happiness }\OperatorTok{~}\StringTok{ }\KeywordTok{dnorm}\NormalTok{( mu , sigma ),}
\NormalTok{        mu <-}\StringTok{ }\NormalTok{a }\OperatorTok{+}\StringTok{ }\NormalTok{bA}\OperatorTok{*}\NormalTok{A,}
\NormalTok{        a }\OperatorTok{~}\StringTok{ }\KeywordTok{dnorm}\NormalTok{( }\DecValTok{0}\NormalTok{ , }\DecValTok{1}\NormalTok{ ),}
\NormalTok{        bA }\OperatorTok{~}\StringTok{ }\KeywordTok{dnorm}\NormalTok{( }\DecValTok{0}\NormalTok{ , }\DecValTok{2}\NormalTok{ ),}
\NormalTok{        sigma }\OperatorTok{~}\StringTok{ }\KeywordTok{dexp}\NormalTok{(}\DecValTok{1}\NormalTok{)}
\NormalTok{    ) , }\DataTypeTok{data=}\NormalTok{d2 )}
\KeywordTok{precis}\NormalTok{(m6}\FloatTok{.10}\NormalTok{)}
\end{Highlighting}
\end{Shaded}

\begin{verbatim}
##                mean         sd       5.5%     94.5%
## a      1.649248e-07 0.07675015 -0.1226614 0.1226617
## bA    -2.728620e-07 0.13225976 -0.2113769 0.2113764
## sigma  1.213188e+00 0.02766080  1.1689803 1.2573949
\end{verbatim}

\begin{Shaded}
\begin{Highlighting}[]
 \KeywordTok{compare}\NormalTok{( m6}\FloatTok{.9}\NormalTok{ , m6}\FloatTok{.10}\NormalTok{ )}
\end{Highlighting}
\end{Shaded}

\begin{verbatim}
##           WAIC       SE    dWAIC      dSE    pWAIC       weight
## m6.9  2713.971 37.54465   0.0000       NA 3.738532 1.000000e+00
## m6.10 3101.906 27.74379 387.9347 35.40032 2.340445 5.768312e-85
\end{verbatim}

\begin{enumerate}
\def\labelenumi{\arabic{enumi}.}
\setcounter{enumi}{2}
\tightlist
\item
  Reconsider the urban fox analysis from last week's homework. Use WAIC
  or LOO based model comparison on five different models, each using
  weight as the outcome, and containing these sets of predictor
  variables:
\end{enumerate}

\begin{enumerate}
\def\labelenumi{(\arabic{enumi})}
\tightlist
\item
  avgfood + groupsize + area
\item
  avgfood + groupsize
\item
  groupsize + area
\item
  avgfood
\item
  area Can you explain the relative differences in WAIC scores, using
  the fox DAG from last week's homework? Be sure to pay attention to the
  standard error of the score differences (dSE).
\end{enumerate}

\begin{Shaded}
\begin{Highlighting}[]
\KeywordTok{library}\NormalTok{(rethinking)}
\KeywordTok{data}\NormalTok{(foxes)}
\NormalTok{d <-}\StringTok{ }\NormalTok{foxes}
\NormalTok{d}\OperatorTok{$}\NormalTok{W <-}\StringTok{ }\KeywordTok{standardize}\NormalTok{(d}\OperatorTok{$}\NormalTok{weight)}
\NormalTok{d}\OperatorTok{$}\NormalTok{A <-}\StringTok{ }\KeywordTok{standardize}\NormalTok{(d}\OperatorTok{$}\NormalTok{area)}
\NormalTok{d}\OperatorTok{$}\NormalTok{F <-}\StringTok{ }\KeywordTok{standardize}\NormalTok{(d}\OperatorTok{$}\NormalTok{avgfood)}
\NormalTok{d}\OperatorTok{$}\NormalTok{G <-}\StringTok{ }\KeywordTok{standardize}\NormalTok{(d}\OperatorTok{$}\NormalTok{groupsize)}
\NormalTok{m1 <-}\StringTok{ }\KeywordTok{quap}\NormalTok{(}
    \KeywordTok{alist}\NormalTok{(}
\NormalTok{        W }\OperatorTok{~}\StringTok{ }\KeywordTok{dnorm}\NormalTok{( mu , sigma ),}
\NormalTok{        mu <-}\StringTok{ }\NormalTok{a }\OperatorTok{+}\StringTok{ }\NormalTok{bF}\OperatorTok{*}\NormalTok{F }\OperatorTok{+}\StringTok{ }\NormalTok{bG}\OperatorTok{*}\NormalTok{G }\OperatorTok{+}\StringTok{ }\NormalTok{bA}\OperatorTok{*}\NormalTok{A,}
\NormalTok{        a }\OperatorTok{~}\StringTok{ }\KeywordTok{dnorm}\NormalTok{(}\DecValTok{0}\NormalTok{,}\FloatTok{0.2}\NormalTok{),}
        \KeywordTok{c}\NormalTok{(bF,bG,bA) }\OperatorTok{~}\StringTok{ }\KeywordTok{dnorm}\NormalTok{(}\DecValTok{0}\NormalTok{,}\FloatTok{0.5}\NormalTok{),}
\NormalTok{        sigma }\OperatorTok{~}\StringTok{ }\KeywordTok{dexp}\NormalTok{(}\DecValTok{1}\NormalTok{)}
\NormalTok{    ), }\DataTypeTok{data=}\NormalTok{d )}
\NormalTok{m2 <-}\StringTok{ }\KeywordTok{quap}\NormalTok{(}
    \KeywordTok{alist}\NormalTok{(}
\NormalTok{        W }\OperatorTok{~}\StringTok{ }\KeywordTok{dnorm}\NormalTok{( mu , sigma ),}
\NormalTok{        mu <-}\StringTok{ }\NormalTok{a }\OperatorTok{+}\StringTok{ }\NormalTok{bF}\OperatorTok{*}\NormalTok{F }\OperatorTok{+}\StringTok{ }\NormalTok{bG}\OperatorTok{*}\NormalTok{G,}
\NormalTok{        a }\OperatorTok{~}\StringTok{ }\KeywordTok{dnorm}\NormalTok{(}\DecValTok{0}\NormalTok{,}\FloatTok{0.2}\NormalTok{),}
        \KeywordTok{c}\NormalTok{(bF,bG) }\OperatorTok{~}\StringTok{ }\KeywordTok{dnorm}\NormalTok{(}\DecValTok{0}\NormalTok{,}\FloatTok{0.5}\NormalTok{),}
\NormalTok{        sigma }\OperatorTok{~}\StringTok{ }\KeywordTok{dexp}\NormalTok{(}\DecValTok{1}\NormalTok{)}
\NormalTok{    ), }\DataTypeTok{data=}\NormalTok{d )}
\NormalTok{m3 <-}\StringTok{ }\KeywordTok{quap}\NormalTok{(}
    \KeywordTok{alist}\NormalTok{(}
\NormalTok{        W }\OperatorTok{~}\StringTok{ }\KeywordTok{dnorm}\NormalTok{( mu , sigma ),}
\NormalTok{        mu <-}\StringTok{ }\NormalTok{a }\OperatorTok{+}\StringTok{ }\NormalTok{bG}\OperatorTok{*}\NormalTok{G }\OperatorTok{+}\StringTok{ }\NormalTok{bA}\OperatorTok{*}\NormalTok{A,}
\NormalTok{        a }\OperatorTok{~}\StringTok{ }\KeywordTok{dnorm}\NormalTok{(}\DecValTok{0}\NormalTok{,}\FloatTok{0.2}\NormalTok{),}
        \KeywordTok{c}\NormalTok{(bG,bA) }\OperatorTok{~}\StringTok{ }\KeywordTok{dnorm}\NormalTok{(}\DecValTok{0}\NormalTok{,}\FloatTok{0.5}\NormalTok{),}
\NormalTok{        sigma }\OperatorTok{~}\StringTok{ }\KeywordTok{dexp}\NormalTok{(}\DecValTok{1}\NormalTok{)}
\NormalTok{    ), }\DataTypeTok{data=}\NormalTok{d )}
\NormalTok{m4 <-}\StringTok{ }\KeywordTok{quap}\NormalTok{(}
    \KeywordTok{alist}\NormalTok{(}
\NormalTok{        W }\OperatorTok{~}\StringTok{ }\KeywordTok{dnorm}\NormalTok{( mu , sigma ),}
\NormalTok{        mu <-}\StringTok{ }\NormalTok{a }\OperatorTok{+}\StringTok{ }\NormalTok{bF}\OperatorTok{*}\NormalTok{F,}
\NormalTok{        a }\OperatorTok{~}\StringTok{ }\KeywordTok{dnorm}\NormalTok{(}\DecValTok{0}\NormalTok{,}\FloatTok{0.2}\NormalTok{),}
\NormalTok{        bF }\OperatorTok{~}\StringTok{ }\KeywordTok{dnorm}\NormalTok{(}\DecValTok{0}\NormalTok{,}\FloatTok{0.5}\NormalTok{),}
\NormalTok{        sigma }\OperatorTok{~}\StringTok{ }\KeywordTok{dexp}\NormalTok{(}\DecValTok{1}\NormalTok{)}
\NormalTok{    ), }\DataTypeTok{data=}\NormalTok{d )}
\NormalTok{m5 <-}\StringTok{ }\KeywordTok{quap}\NormalTok{(}
    \KeywordTok{alist}\NormalTok{(}
\NormalTok{        W }\OperatorTok{~}\StringTok{ }\KeywordTok{dnorm}\NormalTok{( mu , sigma ),}
\NormalTok{        mu <-}\StringTok{ }\NormalTok{a }\OperatorTok{+}\StringTok{ }\NormalTok{bA}\OperatorTok{*}\NormalTok{A,}
\NormalTok{        a }\OperatorTok{~}\StringTok{ }\KeywordTok{dnorm}\NormalTok{(}\DecValTok{0}\NormalTok{,}\FloatTok{0.2}\NormalTok{),}
\NormalTok{        bA }\OperatorTok{~}\StringTok{ }\KeywordTok{dnorm}\NormalTok{(}\DecValTok{0}\NormalTok{,}\FloatTok{0.5}\NormalTok{),}
\NormalTok{        sigma }\OperatorTok{~}\StringTok{ }\KeywordTok{dexp}\NormalTok{(}\DecValTok{1}\NormalTok{)}
\NormalTok{), }\DataTypeTok{data=}\NormalTok{d )}


 \KeywordTok{compare}\NormalTok{( m1 , m2 , m3 , m4 , m5 )}
\end{Highlighting}
\end{Shaded}

\begin{verbatim}
##        WAIC       SE     dWAIC      dSE    pWAIC      weight
## m1 322.8847 16.27783  0.000000       NA 4.656959 0.465363694
## m3 323.8985 15.68240  1.013749 2.899417 3.718565 0.280323674
## m2 324.1284 16.13964  1.243666 3.598475 3.859897 0.249881396
## m4 333.4444 13.78855 10.559695 7.193396 2.426279 0.002370193
## m5 333.7239 13.79447 10.839215 7.242069 2.650636 0.002061043
\end{verbatim}

avgfood is a confounder of groupsize --\textgreater{} weight, so models
that include avgfood and groupsize are causally correct

\begin{enumerate}
\def\labelenumi{(\arabic{enumi})}
\item
  avgfood + groupsize + area: don't really need ara
\item
  avgfood + groupsize
\item
  groupsize + area confounded
\item
  avgfood
\item
  area
\end{enumerate}

\end{document}
